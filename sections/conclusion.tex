\section{Conclusion}
\label{sec:conclusion}

We analyzed the network utilization and performance of three major VCAs under
diverse network conditions and usage modalities. Our results show that VCAs
vary significantly both in terms of their resource utilization and
performance, especially when capacity is constrained.  Differences in behavior
can be attributed to different VCA design parameter values, congestion control
algorithms, and media transport mechanisms. The VCA response also varies
depending on whether constraints exist on the uplink or downlink. Different
behavior often arises due to both differences in encoding strategies, as well
as how the VCAs rely on an intermediate server to deliver video streams and
adapt transmission to changing conditions. Finally, the network utilization of
each VCA can vary significantly depending on the call modality. Somewhat
counterintuitively, calls with more participants can actually reduce any one
participant's upstream utilization, because changes in the video layout on the
user screen ultimately lead to changes in the sent video resolutions. Future
work could extend this analysis to more VCAs, device platforms, and network
impairments. In general, however, performance can begin to degrade at upstream
capacities below 1.5~Mbps, and some VCAs do not compete well with other TCP
streams under constrained settings, suggesting the possible need for the FCC
to reconsider its 25/3 Mbps standard for defining ``broadband''.
