

\section{Network Disruptions}
\label{sec:interruption}

VCAs must handle network disruptions; they may do so in different ways.
Designing to cope with disruptions, such as interruptions to network
connectivity, has become even more pertinent over the past year as users have
increasingly relied on broadband Internet access to use VCAs, which can
sometimes experience connectivity disruptions.  In addition to experiencing
periodic outages, home networks are susceptible to disruptions caused by
temporary congestion along the end-to-end path.

In this section, we aim to understand how VCAs respond to the types of
disruptions that home Internet users might sometimes eperience. We analyze how
VCAs respond to temporary network disruptions during the call by introducing
transient capacity reductions. Using the same setup as in
Figure~\ref{fig:static_setup}, we initiate a five-minute VCA session between
two clients, C1 and C2, both of which are connected to the Internet via a
1~Gbps link. One minute after initiating the call, we reduce the capacity
between C1 and the router for 30 seconds, before reverting back to an
unconstrained link. We conduct two sets of experiments, shaping the uplink in
the first and the downlink in the second. We consider the following shaping
levels: {0.25, 0.5, 0.75, 1.0} Mbps. We only consider disruptions in capacity
to levels of more than 1~Mbps because both Zoom and Meet's average utilization
is below 1~Mbps.

\begin{figure}[t!]
\centering
\begin{subfigure}[t]{.5\textwidth}
    \centering
    \includegraphics[width=.8\textwidth,keepaspectratio]{interrupt/Interrupt-upld.pdf}
    \caption{Average uplink bitrate over time. Grey region indicates period where uplink capacity is constrained to 0.25 Mbps. Vertical dotted lines indicate when the uplink bitrate has returned to the average.}
    \label{fig:ts_upld}
\end{subfigure}\hfill
\begin{subfigure}[t]{.5\textwidth}
      \centering
    \includegraphics[width=.8\textwidth,keepaspectratio]{interrupt/TTR-upld.pdf}
    \caption{The time to recover to average sending bitrate following a drop to the indicated uplink shaping level.}
    \label{fig:TTR_upld}
\end{subfigure}
\caption{VCA response to a 30s drop in available uplink capacity}
\label{fig:interrupt-upld}
\end{figure}

\subsection{Upstream Disruptions}

Figure \ref{fig:ts_upld} illustrates VCA uplink bitrate over the course of a
call. Following a disruption, both the time to recovery and the
characteristics of that recover
differ across VCAs. We quantify the time it takes to
VCA utilization to return to normal by defining a 
\textit{time to recovery} (TTR) metric. We define TTR as the time between when the
interruption ends and when the five-second rolling median bitrate reaches the
median bitrate before interruption, also referred to as nominal bitrate. 

\paragraph{Time to Recovery}: Figure \ref{fig:TTR_upld} shows how extent of
the disruption to upstream connectivity affects each VCA's time to recovery.
The more severe the capacity reduction, the longer the VCAs need to recover.

\teams takes longer to recover even at less severe shaping levels for two
reasons: (1)~the nominal bitrate of Teams is higher than Meet and Teams,
(2)~Teams increases the upstream bitrate slowly immediately after the
interruption before increasing quickly back to normal (as shown in
Figure~\ref{fig:ts_upld}). Meet also observes a similar recovery pattern when
the disruption drops capacity to 0.25~Mbps. However, it recovers much faster
for the case of less severe disruptions, mostly because its nominal bitrate is
around 0.96 Mbps. 

\paragraph{Recovery Patterns}: While Meet and Teams follow a more rapid trend,
Zoom's recovery is different: It takes the longest time to recover under
severe disruptions. According to Figure \ref{fig:ts_upld}, Zoom follows a
stepwise recovery, with an almost-linear increase immediately after
interruption. It then enters into a periodic-probing phase where it increases
the sending rate, stays at it for sometime before increasing it again. The
probing phase continues well above the its nominal bitrate, before finally
reducing the bitrate.  Zoom does not return to a steady-state sending pattern
until two minutes after the disruption, in the meantime sending at much higher
rates. At first glance, such probing might appear to be bad design as it could
introduce additional packet loss and delay harming both Zoom's own performance
and the performance of other applications.  We believe, however, that Zoom may
be using redundant FEC packets to gauge available bandwidth similar to the
FBRA congestion control proposed by Nagy et al.~\cite{nagy2014congestion}.
Thus, even if such behavior induces packet loss, the user performance may not
suffer. Nevertheless, this inefficient use of the uplink, however, could
disturb other applications on the same link, leading to a poor quality of
experience for competing applications. 

\begin{figure}[t!]
 \centering
\begin{subfigure}[t]{.5\textwidth}
   \centering
    \includegraphics[width=.9\textwidth,keepaspectratio]{interrupt/Interrupt-dnld.pdf}
    \caption{Average downlink bitrate over time. Grey region indicates period where downlink capacity is constrained to 0.25 Mbps. Vertical dotted lines indicate when the dowlink bitrate has returned to the average.}
    \label{fig:ts-dnld}
\end{subfigure}
\begin{subfigure}[t]{.5\textwidth}
  \centering
    \includegraphics[width=.9\textwidth,keepaspectratio]{figures/interrupt/TTR-dnld.pdf}
    \caption{The time to recover to average receiving bitrate following a drop to the indicated downlink shaping level.}
    \label{fig:TTR_dnld}
\end{subfigure}
\caption{VCA response to a 30s drop in available downlink capacity.}
\label{fig:interrupt-dnld}
\end{figure}

\subsection{Downstream Disruptions}

Turning now to downlink shaping, it is clear from Figure~\ref{fig:TTR_dnld} that Teams recovers much slower than Meet and Zoom, always taking at least 20 seconds longer to return to the average rate, regardless of the magnitude of the interruption. Furthermore, Meet and Zoom recover much faster in this case compared to uplink interruptions. This can be explained by the way each VCA sends video. In all three VCAs, C1 and C2 communicate through an intermediary server. Thus, the recovery times depend on the congestion control capabilities at the server. 


As mentioned in Section~\ref{sec:static}, Meet uses an encoding technique called \textit{simulcast}, where the client encodes two versions. The server then sends the appropriate version based on the perceived downlink capacity at the receiver. This way, a sender transmitting on an unconstrained link does not have to adjust its sending and can rely on the server to send the appropriate version. The server can then quickly switch between which version to send to the receiver based on the feedback it receives. This quick recovery is clearly illustrated in \ref{fig:interrupt-dnld}, in which Meet returns to its average rate in under 10 seconds, regardless the severity of the interruption.

Similarly, Zoom uses \textit{scalable video coding} when transmitting video~\cite{zoom_encoding}. Instead of sending several versions of varying quality, Zoom sends many hierarchical "layers" that, when superimposed, produce a high quality video. This allows C2 to continue sending uninterrupted even when C1's downlink capacity shrinks. The server can then recover faster by sending additional layers, once the network conditions improve.  


\begin{figure}[t]
    \centering
    \includegraphics[width=0.45\textwidth,keepaspectratio]{../figures/interrupt/Interrupt-sender.pdf}
    \caption{Client 2 (C2) uplink bitrate. Grey region indicates when C1's downlink capacity is reduced to 0.25 Mbps}
    \label{fig:interrupt-sender}
\end{figure}

\begin{figure}[]
   \centering
    \includegraphics[width=0.5\textwidth,keepaspectratio]{methodology/competition-setup.pdf}
    \caption{Setup for competition experiments.}
    \label{fig:competition-setup}
\end{figure}

\begin{comment}
 \begin{figure*}[t]
    \includegraphics[width=\linewidth]{comp/ul_competition_all.pdf}
    \caption{Share of upstream link and VCA bitrates for VCAs in competition with other flows, as a function of downlink bitrate cap.}
	\label{fig:comp_bitrates_ul}
\end{figure*}
\end{comment}


\begin{figure*}[t!]
    \centering
    \begin{subfigure}[t]{.33\textwidth}
        \centering
        \includegraphics[width=1\textwidth]{figures/comp_all/box_plot_meet_ul_0.5_all.pdf}
        \caption{Meet}
        \label{fig:meet_ul_box}
    \end{subfigure}\hfill
    \begin{subfigure}[t]{.33\textwidth}
        \centering
        \includegraphics[width=1\textwidth]{figures/comp_all/box_plot_teams_ul_0.5_all.pdf}
        \caption{Teams}
        \label{fig:teams_ul_box}
    \end{subfigure}
    \begin{subfigure}[t]{.33\textwidth}
        \centering
        \includegraphics[width=1\textwidth]{figures/comp_all/box_plot_zoom_ul_0.5_all.pdf}
        \caption{Zoom}
        \label{fig:zoom_ul_box}
    \end{subfigure}
    \caption{Uplink bitrate of applications in competition with each VCA, with a link shaped symmetrically at 0.5~Mbps.}
    \label{fig:boxplot-upld}
\end{figure*}




While the intermediary server for Zoom and Meet does congestion control, it only acts as a relay for Teams. During a Teams call, C2 will recognize C1 has limited downlink capacity and adjust its behavior, sending at only the bitrate it knows C1 can handle. Once C1 has more available bandwidth, however, C2 must first probe the connection before returning to its average sending rate. Figure \ref{fig:interrupt-sender} illustrates how C2's sending rate does not change during a Meet call, but drops below the shaping threshold during a Teams call, leading to the slow recovery.


In terms of how efficiently each VCA uses the constrained link, Zoom's downlink and uplink bitrate stays at the shaping level while Meet and Teams fall even lower than the shaping level. Zoom's efficient network utilization may be an artifact of its encoding mechanisms, wherein it can more effectively control the encoding parameters (e.g., svc-encoded layers, FPS, resolution etc.) to match the target bitrate.  

\begin{mdframed}[roundcorner=5pt, backgroundcolor=black!10]
\noindent \textbf{Takeaways}: VCAs are slow to recover from reduction to uplink capacity, all requiring over 25 seconds to recover from severe interruptions to 0.25 Mbps. Only Teams is consistently slow to recover from drops in downlink capacity, even following moderate drops to 1 Mbps. This result can be attributed to VCA's media sending mechanism and how  congestion control is implemented at the intermediate server for each VCA. 
\end{mdframed}


