

\section{Network Disruptions}
\label{sec:interruption}
One challenge in VCA design is deciding how to handle network disruptions. This challenge is especially pertinent as many use VCAs on home networks. Home networks are especially susceptible to disruptions caused by temporary congestion along the end-to-end path.

In this section, we analyze how VCAs respond to temporary network disruptions during the call by introducing transient drops in available bandwidth. Using the same setup as Figure \ref{fig:static_setup}, a 5-minute VCA call is initiated between two clients, C1 and C2, both of which sit behind unconstrained links. One minute after initiating the call, we reduce the available bandwidth between C1 and the router for 30 seconds, before reverting back to an unconstrained link. We conduct two sets of experiments, shaping the uplink in the first and the downlink in the second. We consider the following shaping levels: {0.25, 0.5, 0.75, 1.0} Mbps. We don't shape beyond 1.0 Mbps because both Zoom and Meet's average utilization is below 1.0 Mbps.

\begin{figure}[t!]
\centering
\begin{subfigure}[t]{.5\textwidth}
    \centering
    \includegraphics[width=.9\textwidth,keepaspectratio]{interrupt/Interrupt-upld.pdf}
    \caption{Average uplink bitrate over time. Grey region indicates period where uplink capacity is constrained to 0.25 Mbps. Vertical dotted lines indicate when the uplink bitrate has returned to the average.}
    \label{fig:ts_upld}
\end{subfigure}\hfill
\begin{subfigure}[t]{.5\textwidth}
      \centering
    \includegraphics[width=.9\textwidth,keepaspectratio]{interrupt/TTR-upld.pdf}
    \caption{The time to recover to average sending bitrate following a drop to the indicated uplink shaping level.}
    \label{fig:TTR_upld}
\end{subfigure}
\caption{VCA response to a 30s drop in available uplink capacity}
\label{fig:interrupt-upld}
\end{figure}

\subsection{Uplink Disruptions}
Focusing first on uplink shaping, Figure \ref{fig:ts_upld} illustrates VCA uplink bitrate over the course of a call. It is clear that both the path to recovery and the time to recovery following an interruption differ greatly among the VCAs. We quantify the time it takes to VCA utilization to return to normal by introducing a metric called \textit{time to recovery} (TTR). We define TTR as the time between when the interruption ends and when the 5-second rolling median bitrate reaches the median bitrate before interruption, also referred to as nominal bitrate. %In other words, how long it takes the VCA performance to return to normal following the interruption. 

\paragraph{Time to Recovery}: Figure \ref{fig:TTR_upld} shows how the uplink shaping level affects each VCA's time to recovery. There is a clear trend that the more severe the uplink shaping, the longer the VCAs need to recover. This trend, however, is less pronounced for Teams than it is for Meet and Zoom as it takes longer to recover even at less severe shaping levels. This is because of two reasons: i) the nominal bitrate of Teams is higher than Meet and Teams, ii)  Teams increases the uplink bitrate slowly immediately after the interruption before increasing quickly back to normal (see  Figure~\ref{fig:ts_upld}), almost resembling a cubic function. Meet also observes this cubic-like recovery at the 0.25 Mbps shaping level. However, it recovers much faster at other shaping levels, mostly because its nominal bitrate is around 0.96 Mbps. 

\paragraph{Recovery Patterns}: While Meet and Teams seem to follow a cubic-like trendline, Zoom's recovery is markedly different. It takes the longest time to recover under severe interruptions. Looking at Figure \ref{fig:ts_upld}, Zoom follows a stepwise recovery, with an almost-linear increase right after interruption. It then enters into a periodic-probing phase where it increases the sending rate, stays at it for sometime before increasing it again. The probing phase continues well above the its nominal bitrate, before finally dropping back. Zoom does not return to a ``normal'' sending pattern until 2 minutes after the interruption, sending at much higher rates than necessary. At first look, such probing seems like a bad design as it introduces packet loss on the link, hurting Zoom's own performance let alone others. However, we believe that Zoom may be using redundant FEC packets to gauge available bandwidth similar to the FBRA congestion control proposed by Nagy et al.~\cite{nagy2014congestion}. Thus, even if there is packet loss, the user performance does not suffer. This inefficient use of the uplink, however, could disturb other applications on the same link, leading to a poor quality of experience for competing applications. 

\begin{figure}[t!]
 \centering
\begin{subfigure}[t]{.5\textwidth}
   \centering
    \includegraphics[width=.9\textwidth,keepaspectratio]{interrupt/Interrupt-dnld.pdf}
    \caption{Average downlink bitrate over time. Grey region indicates period where downlink capacity is constrained to 0.25 Mbps. Vertical dotted lines indicate when the dowlink bitrate has returned to the average.}
    \label{fig:ts-dnld}
\end{subfigure}
\begin{subfigure}[t]{.5\textwidth}
  \centering
    \includegraphics[width=.9\textwidth,keepaspectratio]{figures/interrupt/TTR-dnld.pdf}
    \caption{The time to recover to average receiving bitrate following a drop to the indicated downlink shaping level.}
    \label{fig:TTR_dnld}
\end{subfigure}
\caption{VCA response to a 30s drop in available downlink capacity.}
\label{fig:interrupt-dnld}
\end{figure}

\subsection{Downlink Disruptions}

Turning now to downlink shaping, it is clear from Figure~\ref{fig:TTR_dnld} that Teams recovers much slower than Meet and Zoom, always taking at least 20 seconds longer to return to the average rate, regardless of the magnitude of the interruption. Furthermore, Meet and Zoom much faster in this case compared to uplink interruptions. This can be explained by the way each VCA sends video. In all three VCAs, C1 and C2 communicate through an intermediary server. Thus, the recovery times depend on the congestion control capabilities at the server. 


As mentioned in Section~\ref{sec:static}, Meet uses an encoding technique called \textit{simulcast}, where the client encodes two versions ~\cite{nistico2020comparative}. The server then sends the appropriate version based on the perceived downlink capacity at the receiver. This way, a sender transmitting on an unconstrained link does not have to adjust its sending and can rely on the server to send the appropriate version. The server can then quickly switch between which version to send to the receiver based on the feedback it receives. This quick recovery is clearly illustrated in \ref{fig:interrupt-dnld}, in which Meet returns to its average rate in under 10 seconds, regardless the severity of the interruption.

Similarly, Zoom uses \textit{scalable video coding} when transmitting video~\cite{zoom_encoding}. Instead of sending several versions of varying quality, Zoom sends many hierarchical "layers" that, when superimposed, produce a high quality video. This allows C2 to continue sending uninterrupted even when C1's downlink capacity shrinks. The server can then recover faster by sending additional layers, once the network conditions improve.  


\begin{figure}[t]
    \centering
    \includegraphics[width=0.45\textwidth,keepaspectratio]{../figures/interrupt/Interrupt-sender.pdf}
    \caption{Client 2 (C2) uplink bitrate. Grey region indicates when C1's downlink capacity is reduced to 0.25 Mbps}
    \label{fig:interrupt-sender}
\end{figure}

\begin{figure}[]
   \centering
    \includegraphics[width=0.5\textwidth,keepaspectratio]{methodology/competition-setup.pdf}
    \caption{Setup for competition experiments.}
    \label{fig:competition-setup}
\end{figure}

 \begin{figure*}[t]
    \includegraphics[width=\linewidth]{comp/ul_competition_all.pdf}
    \caption{Share of upstream link and VCA bitrates for VCAs in competition with other flows, as a function of downlink bitrate cap.}
	\label{fig:comp_bitrates_ul}
\end{figure*}

\begin{figure*}[t!]
    \centering
    \begin{subfigure}[t]{.33\textwidth}
        \centering
        \includegraphics[width=1\textwidth]{figures/comp/box_plot_meet_ul_0.5.pdf}
        \caption{Meet}
        \label{fig:meet_ul_box}
    \end{subfigure}\hfill
    \begin{subfigure}[t]{.33\textwidth}
        \centering
        \includegraphics[width=1\textwidth]{figures/comp/box_plot_teams_ul_0.5.pdf}
        \caption{Teams}
        \label{fig:teams_ul_box}
    \end{subfigure}
    \begin{subfigure}[t]{.33\textwidth}
        \centering
        \includegraphics[width=1\textwidth]{figures/comp/box_plot_zoom_ul_0.5.pdf}
        \caption{Zoom}
        \label{fig:zoom_ul_box}
    \end{subfigure}
    \caption{Uplink bitrate of applications in competition with each VCA, with a link shaped symmetrically at 0.5~Mbps.}
    \label{fig:boxplot-upld}
\end{figure*}




While the intermediary server for Zoom and Meet does congestion control, it only acts as a relay for Teams. During a Teams call, C2 will recognize C1 has limited downlink capacity and adjust its behavior, sending at only the bitrate it knows C1 can handle. Once C1 has more available bandwidth, however, C2 must first probe the connection before returning to its average sending rate. Figure \ref{fig:interrupt-sender} illustrates how C2's sending rate does not change during a Meet call, but drops below the shaping threshold during a Teams call, leading to the slow recovery.


In terms of how efficiently each VCA uses the constrained link, Zoom's downlink and uplink bitrate stays at the shaping level while Meet and Teams fall even lower than the shaping level. Zoom's efficient network utilization may be an artifact of its encoding mechanisms, wherein it can more effectively control the encoding parameters (e.g., svc-encoded layers, FPS, resolution etc.) to match the target bitrate.  

\begin{mdframed}[roundcorner=5pt, backgroundcolor=black!10]
\noindent \textbf{Takeaways}: VCAs are slow to recover from reduction to uplink capacity, all requiring over 25 seconds to recover from severe interruptions to 0.25 Mbps. Only Teams is consistently slow to recover from drops in downlink capacity, even following moderate drops to 1 Mbps. This result can be attributed to how each VCA handles congestion control. Nevertheless, these observations portend lengthy recovery periods for all VCAs on networks with frequent fluctuations in uplink capacity. 
\end{mdframed}


