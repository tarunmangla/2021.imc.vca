\section{Introduction}\label{sec:intro}

Internet users around the world have become increasingly dependent on video
conferencing applications over the past several years, particularly as a
result of the COVID-19 pandemic, which has caused many users to rely almost
exclusively on these applications for remote work, education, healthcare,
social connections, and many other activities.  Some ISPs have reported as
much as a three-fold increase in video conferencing traffic during an
eight-month period in 2020 during the pandemic~\cite{bitag_report}. Our
increased reliance on these video conferencing applications~(VCAs) has highlighted
certain disparities, especially given that nearly 15 million
public school students across the United States during the pandemic lacked
Internet access for remote education. 

As cities and countries around the world moved to close this gap, many asked a
series of simple questions geared towards understanding the level of
connectivity that they needed to provide citizens to guarantee reliable,
high-quality Internet experience: {\em What is the baseline level of Internet
performance needed to support common video conferencing applications for the
activities people commonly use them for (e.g., education, remote work,
telehealth)?} This was the question, in fact, that our own city officials
asked us which motivated us to pursue this question in this paper. The question
has been brought into focus even more over recent months as the United Stated
Federal Communications Commission (FCC) continues to debate the levels of
Internet speed that should classify as a ``broadband'' service offering. The
current standard is 25~Mbps downstream/3~Mbps upstream. With the rise of video
conferencing applications, however, consumer Internet connections saw an
increase in upstream traffic utilization. Some policy advocacy papers have
claimed (without measurements) that the FCC should change its definition
of ``broadband'' to a symmetric 20~Mbps connection on account of these trends.
Moreover, because many users, especially in underserved regions, experience
frequent connectivity disruptions, a comparative analysis of VCAs under
dynamic (and degraded) network conditions can also shed light on the design
practices of VCAs and help identify best design practices.

In light of these discussions and questions from municipal, state, and federal
policymakers, we were motivated to explore the answers to questions concerning
how much network resources video conferencing applications required, how they
responded to network degradations and connectivity interruptions, how they
compete with each other and with other applications, and how these questions
vary depending on the modalities of use (e.g., gallery mode vs. speaker mode).
Similar questions have of course been studied in the
past~\cite{guha2005experimental, baset2004analysis, bonfiglio2008detailed,
bonfiglio2008tracking, xu2012video}, but the vast majority of these studies
are now at least a decade old, during which time both the VCAs themselves and
how we use them has changed dramatically. New VCAs such as Google Meet,
Microsoft Teams, and Zoom have emerged in the last few years alone. In light
of the emergence of these new VCAs and recent drastic shifts in usage pattern
and our dependence on these applications, a re-appraisal of these questions is
timely.

We study the following questions for three popular, modern video conferencing
applications:
\begin{enumerate}
    \itemsep=-1pt
    \item What is the network utilization of common video conferencing
        applications?
    \item What speeds do these VCAs require to deliver good quality of
        experience to users?
    \item How do VCAs respond to temporary disruptions to connectivity?
        and how quickly do they recover when connectivity is restored?
    \item How do VCAs respond in the presence of competing traffic? How do
        they compete with themselves, with other VCAs, and with other
        applications?
    \item How do different usage modalities (number of participants, speaker
        vs. gallery viewing) affect network utilization?
\end{enumerate}
\noindent
We explore these questions by performing controlled laboratory experiments
with emulated network conditions across a collection of clients, collecting
application performance data using provided application APIs. We collect
application data under a variety of network conditions and call modalities,
including different numbers of participants and viewing modes. 
The collected data show that the answers to
the questions differ substantially depending on the VCA. Many of the answers
to these questions surprised us; as we will discover,
sometimes the answers depend quite a lot on the particular video
conferencing application. Table~\ref{tab:results} summarizes our main findings
and where they can be found in the paper. In particular, we found significant
differences in network utilization, encoding strategy, and recovery time after
network disruption. We also discovered that VCAs share available network
resources with other applications differently. We confirm earlier findings 
\cite{cable-labs-vca}, that one participant's choice of how to interact 
in the video conference can affect the network utilization of {\em other} participants.

Our findings have broad implications, for network management, as well as for
policy. For network management, our findings concerning network utilization
and performance under various network conditions have implications for network
provisioning, as access ISPs seek to provision home networks to achieve good
performance for consumers. From a policy perspective, our results are
particularly important: questions about the throughput needed to support
quality video conferencing, especially in multi-user households, was a
question from city government officials that motivated this paper in the first
place. Looking ahead, as city, state, and federal governments both subsidize
broadband connectivity and build out new infrastructure, understanding how
these increasingly popular video conferencing applications consume and share
network resources is of critical importance.


\begin{table}[t]
    \begin{small}
    \begin{tabular}{|p{3.25in}|}
 \hline 
    The average network utilization on an unconstrained link ranges from 0.8
        Mbps to 1.9 Mbps (\cref{subsec:network_utilization}).\\ \\
    Despite using the same WebRTC API, Meet and Teams browser clients differ significantly in how they encode video (e.g., frames per second, resolution) under reduced link capacity (\cref{subsec:application_performance}).
\\ \\
    Recovery times following transient drops in bandwidth are quite different
        and depend on both VCA's congestion control mechanism and the
        direction of the drop, i.e. uplink or downlink. However, all VCAs take
        at least 20 seconds to recover from severe uplink drops to 0.25 Mbps
        (\cref{sec:interruption}).\\\\
    Differences in proprietary congestion control also create unfair bandwidth
        allocations in constrained bandwidth settings. Zoom and Meet can
        consume over $75\%$ of the available downlink bandwidth when competing
        with Teams or a TCP flow (\cref{sec:competition}).\\\\
    Each participant's video layout impacts its own and others' network utilization. Pinning a user's video to the screen (speaker mode) leads to an increase in the user's uplink utilization (\cref{sec:usage_modality}). 
    \\ \hline
\end{tabular}
    \end{small}
    \caption{Main results, and where they can be found in the paper.\label{tab:results}}
\end{table}
\if 0
VCAs are unique among Internet applications in terms of their network resource
requirements. This is because video conferencing is \textit{real-time} and has
\textit{significant uplink usage}. The former makes VCA more sensitive to
network variations compared to other applications. The latter implies
re-thinking of network provisioning, especially home networks which are
provisioned asymmetrically with significantly lower uplink bandwidth. In
addition, VCAs also differ significantly from each other in terms of their
design including user interface, video codecs, and transport mechanisms (see
Section~\ref{sec:background}). The design diversity potentially leads to
differences in their network resource requirements and performance.     
\fi
