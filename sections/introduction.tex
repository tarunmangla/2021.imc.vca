\section{Introduction}\label{sec:intro}
COVID-19 has made video conferencing applications (VCAs) both common and necessary. People now regularly use VCAs for work, school, and telehealth, as activities have shifted from in-person to online. This migration is also reflected in the the increased contribution of VCAs to network traffic, with some internet service providers seeing video conferencing traffic triple over an eight-month period in 2020~\cite{bitag_report}. While perhaps not continuing at pandemic-level usage, the success of remote communication nevertheless indicates that VCAs will continue to play a critical role during and after the pandemic.



%% NEEDS FIXING: We want to say that VCAs are unique in terms of their network resource requirements and there is diversity among VCAs due to differences in their application design. 
VCAs are unique among internet applications in terms of their network resource requirements. This is because video conferencing is \textit{real-time} and has \textit{significant uplink usage}. The former makes VCA more sensitive to network variations compared to other applications. The latter implies re-thinking of network provisioning, especially home networks which are provisioned asymmetrically with significantly lower uplink bandwidth. In addition, VCAs also differ significantly from each other in terms of their design including user interface, video codecs, and transport mechanisms (see Section~\ref{sec:background}). The design diversity can and does lead to differences in their network resource requirements and performance.     %This leads to different VCA requirements in compars compared to other popular applications, say on-demand video streaming which mostly has only downlink traffic and a video buffer to mitigate short-term network variations. 
% Furthermore, there are a variety of competing VCAs in use today. Not only do these VCAs differ in their application-layer design (e.g., user features, video codecs) but also transport mechanisms. This is because most VCAs rely on UDP for transport and, thus, implement essential transport functions, i.e., congestion control, in the upper layers. %Based on these differences, a good end-user VCA experience depends both on VCA design and provisioning of the network.

 % Despite ostensible similarities, video conferencing differs from over-the-top video streaming in several ways. Most notably, video conferencing is real-time. Thus, VCAs do not benefit from network optimizations (e.g., buffering) used in streaming applications and require different strategies to adapt to dynamic conditions over ``best-effort" networks. Furthermore, VCAs are characterized by significant uplink utilization compared to video streaming, which require appropriate network provisioning by the ISPs. 
   % [receive the bulk of their traffic and send very little]. VCAs, however, send a video stream on the uplink.   % Video streaming applications can "buffer" by fetching video ahead of time, then sporadically downloading when necessary. VCAs can not download video ahead of time because the video occurs in real-time. 
%As a result, VCAs require a consistent and low-latency connection.

Given VCAs' growing importance, it becomes critical to compare and quantify the resource requirements of different VCAs and how these VCAs perform under different network conditions. This can provide insights into the minimum  ``speed'', or bandwidth, requirements the VCA needs to deliver a high quality of experience. Because people continue to rely on using VCAs on home networks, understanding these requirements will inform the bandwidth recommendations of policymakers like the FCC and also guide ISPs in the provisioning of access links. A comparative analysis of VCAs under dynamic and especially poor network conditions can also shed light on the design practices of VCAs and help identify best design practices.

% earliest studies on VCAs focused on Skype, the dominant application in the VCA landscape at the time. These works sought to understand Skype from both a design and performance perspective. As the VCA landscape evolved over time, later work shifted their attention to newer VCAs and the protocols they used. 
While significant research and measurement studies have been conducted on VCAs in the past~\cite{guha2005experimental, baset2004analysis, bonfiglio2008detailed, bonfiglio2008tracking, xu2012video}, both the VCAs and how we use them has since changed. Several new VCAs (e.g. Microsoft Teams, Google Meet) have been introduced, while existing VCAs (e.g. Zoom) have undergone a renewal. In light of these developments, our work provides a fresh take on the resource requirements and performance of modern VCAs. More specifically, we conduct a comparative analysis of performance and network utilization of three popular VCAs: Google Meet, Microsoft Teams, and Zoom. %We can determine these requirements by measuring VCA performance and consumption under realistic network conditions, including low bandwidth availability and other activity on the same network.
Our goal in this study is to understand the following:
\begin{enumerate}[noitemsep]
    \item How VCA performance varies under different link capacities.
    \item How VCAs respond to temporary drops in bandwidth availability.
    \item How VCAs interact with competing applications on a constrained link.
    \item How usage modalities (number of participants, speaker vs. gallery viewing) affect network consumption.
\end{enumerate}

 % Of course, home network conditions are not fixed, nor is there usually only one application running. In studying how VCAs respond to network interruptions and competing applications on same link, we aim to recreate the dynamic network conditions of a home network.

We answer these questions by a series of controlled experiments collecting VCA performance data under a variety of network conditions and call modalities (e.g., number of participants). The collected data show that the answers to the questions differ substantially depending on the VCA. The main findings are summarized below:% 
\begin{itemize}[noitemsep]
    \item The average network utilization on an unconstrained link ranges from 0.8 Mbps to 1.9 Mbps. 
    \item Despite using the same WebRTC API, Meet and Teams differ significantly in how they adapt video encoding parameters (e.g. FPS, resolution, quantization) to reduced link capacity.  
    \item Recovery times following transient drops in bandwidth are quite different and depend on both VCA's congestion control mechanism and the direction of the drop, i.e. uplink or downlink. However, all VCAs take at least 20 seconds to recover from severe uplink drops to 0.25 Mbps. 
    \item Differences in proprietary congestion control also create unfair bandwidth allocations in constrained bandwidth settings. Zoom and Meet can consume over $75\%$ of the available downlink bandwidth when competing with Teams or a TCP flow.
    \item Each participant's video layout impacts its own and others' network utilization. Pinning a user's video to the screen (speaker mode) leads to an increase in the user's uplink utilization.
\end{itemize}

% \begin{itemize}
% \item VCAs have been touted as a killer application for the Internet connecting millions of users. Significant research and measurement studies have been conducted benchmarking the VCAs of the time, their design choices and its impact on application performance. 

% \item The VCA application has undergone significant change in the covid-era of work from home/school with either new VCAs have been introduced or existing applications significantly revamped.

% \item In the context of these changes, it is important to revist the earlier measurement studies especially understanding how these changes relate to application performance and network consumption.

% \item In this paper we do a comparative analysis of three popular applications, namely zoom, meets, and teams. 

% \item Our analysis compares the VCAs along these three questions: i) How does the application performance vary under different network conditions?, ii) What is the impact of different VCA usage modality on network consumption, iii) Are the VCA applications fair in terms of bandwidth sharing when compared to other applications or transport protocols.

% \item Underlying question: what network conditions are required to support multiple streams through a link, at full performance?  Conversely, what network conditions are associated with degraded performance?  (Implications for broadband debates.)

% \item Insights: We find that ... 
% \end{itemize}
