\section{Introduction}\label{sec:intro}
COVID-19 has made video conferencing applications (VCAs) both common and necessary. People now regularly use VCAs for work, school, and telehealth, as activities have shifted from in-person to online. The success of remote communication indicates that VCAs will continue to play a critical role during and after the pandemic. However, VCAs are unique among internet applications and their increased use has altered the makeup of internet traffic. 

VCAs differ from other popular types of internet activity in several ways. Most notably, VCAs have far greater uplink utilization. Video streaming and internet browsing, two dominant sources of internet traffic, [receive the bulk of their traffic and send very little]. VCAs, however, send a video stream on the uplink. Further, VCAs do not benefit from network optimizations used in video streaming. Video streaming applications can "buffer" by fetching video ahead of time, then sporadically downloading when necessary. VCAs can not download video ahead of time because the video occurs in real-time. As a result, VCAs require a consistent and low-latency connection.

The earliest studies on VCAs focused on Skype, the dominant application in the VCA landscape at the time. These works sought to understand Skype from both a design and performance perspective. As the VCA landscape evolved over time, later work shifted their attention to newer VCAs and the protocols they used. 

While significant research and measurement studies have been conducted on VCAs in the past, both the VCAs and how we use them has since changed. Several new VCAs (e.g. Microsoft Teams, Google Meet) have been introduced, while existing VCAs (e.g. Zoom) have undergone a renewal. In light of these developments, it is important to investigate how differing VCA design choices impact application performance and network consumption. 


The continuing reliance on VCAs makes understanding these network requirements precisely crucial. In this paper, we conduct a comparative analysis of three popular VCAs: Google Meet, Microsoft Teams, and Zoom. We can determine these requirements by measuring VCA performance and consumption under realistic network conditions, including low bandwidth availability and other activity on the same network. Our goal is to understand the following:
\begin{enumerate}
    \item How VCAs performance varies under different static network conditions.
    \item How VCAs respond to temporary drops in bandwidth availability.
    \item How VCAs interact with competing applications on a constrained link.
    \item How usage modalities (number of participants, speaker vs. gallery viewing) affect network consumption.
\end{enumerate}

Answering the first question gives us insight into the minimum network requirements for each VCAs. 




% \begin{itemize}
% \item VCAs have been touted as a killer application for the Internet connecting millions of users. Significant research and measurement studies have been conducted benchmarking the VCAs of the time, their design choices and its impact on application performance. 

% \item The VCA application has undergone significant change in the covid-era of work from home/school with either new VCAs have been introduced or existing applications significantly revamped.

% \item In the context of these changes, it is important to revist the earlier measurement studies especially understanding how these changes relate to application performance and network consumption.

% \item In this paper we do a comparative analysis of three popular applications, namely zoom, meets, and teams. 

% \item Our analysis compares the VCAs along these three questions: i) How does the application performance vary under different network conditions?, ii) What is the impact of different VCA usage modality on network consumption, iii) Are the VCA applications fair in terms of bandwidth sharing when compared to other applications or transport protocols.

% \item Underlying question: what network conditions are required to support multiple streams through a link, at full performance?  Conversely, what network conditions are associated with degraded performance?  (Implications for broadband debates.)

% \item Insights: We find that ... 
% \end{itemize}
