\section{Approach Overview}\label{sec:approach}

The experiments are conducted in a lab environment but aim to simulate realistic VCA usage. To that end, we use the following setup throughout the experiments.

\textbf{Hardware}: All data is collected on two identical Dell Latitude 3300 laptops running 20.04.1 Ubuntu. The laptops have a resolution of the laptops is 1366 $\times$ 768 pixels. The laptops have a wired connection to a Turris Omnia 2020 router and access an unconstrained gigabit link on the University of Chicago network. 

\textbf{Automating Calls}: It is crucial that the automated calls are as close to an actual call as possible, as any deviation from real call use may warp results. We take several steps to recreate the in-call process.

All VCAs are tested in their native client unless otherwise indicated. Note, the application's "native client" refers to the Chrome browser for Meet and the desktop applications for Zoom and Teams. In the first set of experiments, we compare browser vs. client utilization for Teams and Zoom. In-browser tests will be specified by \textit{Teams-chrome} or \textit{Zoom-chrome}. 

We use the PyAutoGui and AutoGui Python packages to automate joining and leaving calls. We also use Selenium browser to bypass the CAPTCHA before joining a Zoom call in Chrome browser. In this case, Selenium is not run headless, but is run exactly as it would be in Chrome browser. All experiments are conducted with the laptop lid open and the application window maximized. Run otherwise, the applications may adapt their behavior. For example, if the application window is minimized or the lid is closed, the download bitrate may decrease.

Finally, we use a 1280 $\times$ 720 pre-recorded talking-head video as the video source during calls to both replicate a real video call and ensure consistency across experiments. Using the webcam feed is a non-starter because without movement, the VCAs compress the video and ultimately send at a much lower rate than during a normal call. 

\textbf{Network Shaping}: We simulate network conditions by configuring traffic control using \texttt{tc} on the laptop and router for uplink and downlink shaping, respectively. During the competition experiments, all shaping occurs at the router.

\textbf{Inter-Laptop Communication}: The workflow is controlled with by a wrapper script on laptop 1. Laptop 1 establishes a socket connection with laptop 2 to control automation on laptop 2.

\textbf{Competing Flows}: 
