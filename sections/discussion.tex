\section{Limitations and Future Work}
\label{sec:discussion}

\textbf{Generalizability to other VCA contexts}: While, our work focuses on three well-known desktop applications, it is important to measure performance of other popular VCAs across different device platforms. Our framework, using PyAutoGUI for automation, can be easily extended to other VCAs. Moreover, it can also work on other device platforms using Linux, MacOS or Windows. We can also include other network profiles that represent more dynamic WiFi and cellular contexts. We plan to release our code so that it can be used by other for extending the experiments to these other contexts.

% This work can be extended by investigating how other popular VCAs behave in our experiments.  , mobile video conferencing apps like Facetime, WhatsApp, and Facebook Messenger are immensely popular. Used primarily on Wifi and LTE, mobile VCAs use connections that experience far more dynamic network conditions than do desktop VCAs. 

\textbf{Application performance metrics}: We mostly focus on the network layer VCA performance. Analyzing the application performance statistics can shed more light on the VCA behavior and user quality of experience, as is also seen through a subset of our results using WebRTC statistics in Section~\ref{subsec:application_performance}. However, it can be challenging to obtain application performance metrics for all VCAs, especially the native clients. To better compare application performance across applications, a more robust measurement approach is needed. Future work will explore the methodologies proposed in the related work, such as using annotated videos~\cite{xu2012video} or network traffic captures ~\cite{dasari2018scalable}, to infer application metrics.

% In our future work, we  can be differences We use WebRTC statistics to obtain application performance metrics for Meet and Teams when used in Chrome. Given the performance degradation seen between the native client and the Chrome versions of Teams and Zoom, it is of interest to obtain the same performance metrics for calls had in client. While some call stats are available via the Zoom API, they differ from metrics provided by WebRTC in which metrics are collected, how the metrics are defined, and the granularity at which the measurements are taken. 

\textbf{Additional network conditions}: While we consider only limitations in bandwidth to emulate poor network conditions, other network factors such as latency, packet loss, and jitter could affect application performance and client utilization. Further, home networks can experience changing network conditions that can impact more than one of these factors. It is important to conduct future in-lab experiments that vary these additional network parameters. Of course, these experiments focus only on issues in the ``last-mile'' when there could be problems anywhere along the end-to-end path. However, emulating all kinds of impairments is challenging to accomplish in-lab. An alternative could be to collect VCA performance metrics from a large number of users in the real world. 