\section{Future Work}
\label{sec:discussion}

\paragraph{Generalizability to other VCA contexts}: While this paper focuses on
three popular video conferencing applications, the methods in this paper could
be used to measure utilization and performance of other VCAs, across different
device platforms. Our framework, using PyAutoGUI for automation, can be
applied to other VCAs. It can also work on device platforms based on Linux,
MacOS, or Windows. We can also include other network profiles that represent
other contexts, such as WiFi and cellular. We plan to release all of our
experiment data and code so that it others can extend this work 
to other contexts.


\paragraph{Application performance metrics}: Analyzing application performance
statistics can shed more light on the VCA behavior and user quality of
experience, as is also seen through a subset of our results using WebRTC
statistics in Section~\ref{subsec:application_performance}. It is challenging
to obtain application performance metrics for all VCAs, especially native
clients. Future research could explore the methods from related work, such as
using annotated videos~\cite{xu2012video} or network traffic
captures~\cite{dasari2018scalable} to infer application metrics.

\paragraph{Performance under other network conditions}: Other network factors
such as latency, packet loss, and jitter could affect VCA performance and
utilization.  Future work could explore the effects of these parameters.
Experiments in this paper also focus on issues in the last-mile access
network, when there could be problems anywhere along the end-to-end path.
However, emulating all kinds of impairments is challenging to accomplish
in-lab. An alternative could be to collect VCA performance metrics from a
large number of users in the real world. 

\paragraph{Further exploration of VCA design}: Some of our results reveal
unusual VCA behavior that we found difficult to explain and deserve further
exploration. Examples include: (1)~Some VCAs exhibit different behavior
depending on if the competition exists in the upstream or downstream
direction, (2)~\teams utilization is not consistent with the video layout for
different call modalities; and (3)~\meet shows unusual encoding behavior
(specifically, increased frame rates) when network capacity is very low.
