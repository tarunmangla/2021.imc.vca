\section{Limitations and Future Work}
\label{sec:discussion}

\paragraph{Generalizability to other VCA contexts}: 
\rev{}{This paper focuses on understanding performance and utilization of three popular video conferencing applications mostly over a Linux-based setup in a controlled environment. Clearly, the real-world VCA performance is impacted by a variety of multiple factors including the end-user location, device platform, access network technology, the VCA backbone infrastructure, and streaming mechanisms (e.g., encoding, bitrate adaptation). While this study is a first step in isolating the impact of some of these individual factors on VCA performance, a continued evaluation over a variety of streaming contexts is required for a thorough understanding, especially as the VCAs evolve over time. We believe the methods in this paper could
be extended to other VCAs and across different
device platforms. For instance, our automation framework using PyAutoGUI can be
applied to other VCAs and other desktop device platforms including
MacOS and Windows. We plan to release our
experiment data and code so that others can extend this work 
to these contexts.} 

\rev{}{Finally, an increasing amount of VCAs are being used over mobile devices which are characterized by significant differences compared to the desktop VCAs, such as differences in mobile application architecture and network conditions especially if the access technology is cellular. We believe similar automation frameworks need to be developed for mobile devices to draw conclusions for these platforms.}

\begin{comment}
While this paper focuses on
three popular video conferencing applications, the methods in this paper  \rev{}{Although we conduct the majority of our experiments on 
a Linux-based machine, most users have either a Mac or Windows device.}
Furthermore, \rev{we can include other network profiles that represent
other contexts, such as WiFi and cellular}{we can not draw conclusions about how
VCAs perform on mobile devices because the mobile app architecture is different
from the desktop version and mobile devices connect primarily over Wi-Fi or 
cellular}. \rev{}{The code used for our experiments and our data will be 
publicly available so that other can extend this work to other contexts}.
\end{comment}

\paragraph{Application performance metrics}: Analyzing application performance
statistics can shed more light on the VCA behavior and user quality of
experience, as is also seen through a subset of our results using WebRTC
statistics in Section~\ref{subsec:application_performance}. It is challenging
to obtain application performance metrics for all VCAs, especially native
clients. Future research could explore the methods from related work, such as
using annotated videos~\cite{xu2012video} or network traffic
captures~\cite{dasari2018scalable} to infer application metrics.

\paragraph{Performance under other network conditions}: Other network factors
such as latency, packet loss, and jitter could affect VCA performance and
utilization.  Future work could explore the effects of these parameters,
  \rev{}{or simulate network conditions realistic on wireless connections}.
Experiments in this paper also focus on issues in the last-mile access
network, when there could be problems anywhere along the end-to-end path.
\rev{}{Our work focuses on isolating the conditions under which VCAs fail
by conducting experiments in-lab where we can control for network 
factors such as cross-traffic, jitter, and packet loss better than in the wild. 
With this step complete, it is important to collect VCA performance metrics 
from a large number of users
in real networks and compare the results. Because real-world network conditions
are more complex than in-lab, we can not claim that our results are representative
in all cases. However, they are an important step in the overall evaluation of VCAs
and lay the groundwork for future studies.}
\rev{However, emulating all kinds of impairments is challenging to accomplish
in-lab. An alternative could be to collect VCA performance metrics from a
large number of users in the real world.}{}

\paragraph{Further exploration of VCA design}: Some of our results reveal
unusual VCA behavior that we found difficult to explain and deserve further
exploration. Examples include: (1)~Some VCAs exhibit different behavior
depending on if the competition exists in the upstream or downstream
direction, (2)~\teams utilization is not consistent with the video layout for
different call modalities; and (3)~\meet shows unusual encoding behavior
(specifically, increased frame rates) when network capacity is very low.
