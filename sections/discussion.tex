\section{Limitations and Future Work}
\label{sec:discussion}

\paragraph{Generalizability to other VCA contexts}: 
\rev{}{This paper focuses on understanding the performance and network utilization of three popular video conferencing applications mostly over a Linux-based setup in a controlled environment. Clearly, real-world VCA performance is impacted by a variety of factors including end-user location, device platform, access network technology, the VCA backbone infrastructure, and streaming mechanisms (e.g., encoding or bitrate adaptation). 
In addition, Figure ~\ref{subfig:uplink_browser} shows that Teams has higher utilization in the native application than in browser. This indicates that VCA performance in web clients may not generalize to the native applications.
While this study is a first step in isolating the impact of some of these individual factors on VCA performance, a continued evaluation over a variety of streaming contexts is required for a thorough understanding, especially as the VCAs evolve over time. We believe the methods in this paper could
be extended to other VCAs and across different
device platforms. For instance, our experiment automation framework using PyAutoGUI can be easily
extended to other VCAs and other desktop device platforms including
\textit{macOS} and \textit{Microsoft Windows}. We demonstrate this by repeating the experiments in Section~\ref{sec:static} for macOS 
on Meet, Zoom, and Teams-Chrome (see Appendix~\ref{appendix:static}). We have made our
experiment code publicly available so that others can extend this work 
to these contexts~\cite{vca_code}.} 

\rev{}{Finally, VCAs are increasingly used on mobile devices which differ from laptop or desktop clients in terms of operating systems, constrained resources, and even network conditions in some cases  (e.g., while using cellular network). We believe similar automation frameworks need to be developed for mobile devices to draw conclusions for these platforms.}

\begin{comment}
While this paper focuses on
three popular video conferencing applications, the methods in this paper  \rev{}{Although we conduct the majority of our experiments on 
a Linux-based machine, most users have either a Mac or Windows device.}
Furthermore, \rev{we can include other network profiles that represent
other contexts, such as WiFi and cellular}{we can not draw conclusions about how
VCAs perform on mobile devices because the mobile app architecture is different
from the desktop version and mobile devices connect primarily over Wi-Fi or 
cellular}. \rev{}{The code used for our experiments and our data will be 
publicly available so that other can extend this work to other contexts}.
\end{comment}

\paragraph{Application performance metrics}: Analyzing application performance
statistics, \rev{}{especially the audio and the video metrics individually,} can shed more light on the VCA behavior and user quality of
experience. \rev{}{While we demonstrate the impact of network impairments on video performance metrics through a subset of our results using WebRTC
statistics in Section~\ref{subsec:application_performance}}, it is challenging
to obtain application performance metrics for all VCAs, especially native
clients.  Future research could explore the methods from related work, such as
using annotated \rev{}{audio and} video~\cite{xu2012video} or network traffic
captures~\cite{dasari2018scalable, mangla2018emimic} to infer application metrics.

\paragraph{Performance under other network conditions}: 
\rev{}{Our work mainly focuses on understanding the impact of throughput-based impairments on VCA performance.  We have conducted experiments in-lab, where we can control for other network
factors such as latency, jitter, and packet loss. 
With this step complete, it is important to study the impact of these other network factors. For instance, future work could examine the impact of variations in the packet loss or latency over the congestion control for different VCAs.}

\rev{}{Finally, we acknowledge that real-world networks are more complex and dynamic than in-lab networks. Our results, therefore, are not necessarily representative in all cases. At the same time, it can be challenging to emulate real-world networks in a controlled environment. A future study could collect VCA performance metrics and network metrics from real-world users. 
Alternatively, these analyses could be performed through aggregate traffic 
  on enterprise networks (e.g., university networks).
These organizations could also leverage the consoles or APIs provided with most enterprise VCA subscriptions.}  % they are an important first step in the overall evaluation of the modern VCAs and lay the groundwork for future studies. For instance, future studies could 

\begin{comment}
collect VCA performance metrics 
from a large number of users
in real networks and compare the results. Because real-world network conditions
are more complex than in-lab, we can not claim that our in-lab results are representative
in all cases. However, they are still an important step in the overall evaluation of VCAs
and lay the groundwork for future studies.
Experiments in this paper also focus on issues in the last-mile access
network, when there could be problems anywhere along the end-to-end path.
Future work could explore the effects of these parameters,
  \rev{}{or simulate network conditions realistic on wireless connections}.
\rev{However, emulating all kinds of impairments is challenging to accomplish
in-lab. An alternative could be to collect VCA performance metrics }{}
\end{comment}

\paragraph{Further exploration of VCA design}: Some of our results reveal
unusual VCA behavior that we found difficult to explain and deserve further
exploration. Examples include: (1)~Some VCAs exhibit different behavior
depending on if the competition exists in the upstream or downstream
direction, (2)~\teams' network utilization does not follow a clear pattern,
with respect to the number of participants or their video layout; and (3)~\meet shows unusual encoding behavior
(specifically, increased frame rates) when network capacity is very low.
