\abstract{
% Video conferencing has been one of the killer applications of the Internet. The video conferencing application (VCA) landscape has recently undergone significant change owing to the ongoing covid pandemic with the introduction of new VCA applications (e.g., Google Meet, Microsoft Teams) or revamping of incumbent VCA (e.g., Zoom, Bluejeans) to facilitate user engagement. The VCAs, however, have differed in their design choices and features. Understanding these differences with their impact on application performance and network consumptions can help in improving the design of future VCAs.  

% In this paper, we conduct extensive measurments contrasting the design differences for three popular VCAs, namely Zoom, Google Meet, and Microsoft Teams. We first study the application performance metrics in a 2-person call under different network conditions. We find that differences in application performance under same network conditions which can be attributed to differences in encoding mechanisms, application-level capping, bandwidth estimation techniques. We also found that performance varies across device platform even for the same application (e.g., native client better than the web client). We next study the impact of usage modality (e.g., gallery vs speaker mode in zoom) on network consumption and find some interesting differences in network consumption based on the usage modality. For instance, Zoom typically increases the sent video resolution of the user under speaker mode. Finally, we also study how different VCAs share bandwidth with other applications (file download, video streaming, and other VCAs) in terms of fairness. We find interesting differences wherein we find that applications do not share bandwidth fairly with other internet applications. 
% %A systematic understanding of the VCA design and its implications on network-levek performance is can be useful especially for improving the design of these services. In this paper, we conduct extensive measurements to understand and contrast the design of three popular VCAs, namely Zoom, Google Meet, and Microsoft Teams. 



The VCA landscape has changed significantly because of COVID-19, with the introduction of new VCAs (e.g. Google Meet, Microsoft Teams) and renewal of incumbent VCAs (e.g. Zoom, Bluejeans). Because these VCAs differ in their design choices and features, it is critical that we understand how these differences impact application performance and network consumption under different network conditions and call modalities. This has two-fold advantages: i) a comparative analysis can uncover design issues and best practices that can in turn be used to improve current and future VCAs, and ii) understanding VCA network utilization under different contexts can inform ISPs with network provisioning and telecommunication policy makers in defining Broadband access appropriately.   

In this paper, we conduct extensive measurements contrasting the design difference among three popular VCAs: Zoom, Google Meet, and Microsoft Teams. We find that the VCAs behave quite differently under the same network conditions. [INSERT KEY RESULT FROM STATIC EXPERIMENTS]. Next, we [study] how each VCA responds to dynamic network conditions by introducing (1) temporary bandwidth drops or (2) a competing application (e.g. Netflix, file download) on the same link. The VCAs do not recover efficiently from bandwidth interruptions. All require at least 20 seconds to return to their nominal sending rate following a temporary uplink constraint to 0.25Mbps. When competing with other applications on a constrained link, Zoom aggressively consumes available bandwidth, while Meet and Teams back off at low bandwidths. Finally, we investigate how usage modality (e.g. number of participants, speaker vs. gallery mode) affect consumption. Increasing the number of participants may reduce a user's network utilization. 
We find that a user's sending bitrate at least doubles when their video is pinned by all other participants. 
}
