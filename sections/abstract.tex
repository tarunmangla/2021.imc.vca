\abstract{
Video conferencing applications (VCAs) have become a critical Internet
application, even more so during the COVID-19 pandemic, as users worldwide now rely on them for
work, school, and telehealth. % COVID-19 has spurred unprecedented
%growth in the use of VCAs, with some Internet service providers seeing
%video conferencing traffic triple over an eight-month period
%in 2020. 
It is important to understand the resource requirements of different
VCAs and how these VCAs perform under different network conditions, including:
how much ``speed'' (specifically, upstream and downstream throughput) a VCA needs to support high quality of experience, how
VCAs perform under temporary reductions in available capacity; how they compete
with themselves, with each other, and with
other applications; and how usage modality (e.g., number of participants)
affects utilization.  We study three modern VCAs: Zoom, Google Meet,
and Microsoft Teams.   Answers to these questions differ substantially
depending on VCA.  First, the average utilization on an unconstrained
link varies between 0.8~Mbps and 1.9 Mbps.  Given temporary reduction of
capacity, some VCAs can take as long as 45 seconds to recover to average.  Differences in proprietary
congestion control also create unfair bandwidth allocations: in
constrained bandwidth settings, one Zoom video conference can consume more than 75\%
of the available bandwidth when competing with another VCA (e.g., Meet, Teams).  For some VCAs, client utilization can
decrease as the number of participants increases, due to the reduced video
resolution of each participant's video stream given a larger number of
participants. 
}



% Video conferencing has been one of the killer applications of the Internet. The video conferencing application (VCA) landscape has recently undergone significant change owing to the ongoing covid pandemic with the introduction of new VCA applications (e.g., Google Meet, Microsoft Teams) or revamping of incumbent VCA (e.g., Zoom, Bluejeans) to facilitate user engagement. The VCAs, however, have differed in their design choices and features. Understanding these differences with their impact on application performance and network consumptions can help in improving the design of future VCAs.  

% In this paper, we conduct extensive measurments contrasting the design differences for three popular VCAs, namely Zoom, Google Meet, and Microsoft Teams. We first study the application performance metrics in a 2-person call under different network conditions. We find that differences in application performance under same network conditions which can be attributed to differences in encoding mechanisms, application-level capping, bandwidth estimation techniques. We also found that performance varies across device platform even for the same application (e.g., native client better than the web client). We next study the impact of usage modality (e.g., gallery vs speaker mode in zoom) on network consumption and find some interesting differences in network consumption based on the usage modality. For instance, Zoom typically increases the sent video resolution of the user under speaker mode. Finally, we also study how different VCAs share bandwidth with other applications (file download, video streaming, and other VCAs) in terms of fairness. We find interesting differences wherein we find that applications do not share bandwidth fairly with other internet applications. 
% %A systematic understanding of the VCA design and its implications on network-levek performance is can be useful especially for improving the design of these services. In this paper, we conduct extensive measurements to understand and contrast the design of three popular VCAs, namely Zoom, Google Meet, and Microsoft Teams. 



%The video conferencing application (VCA) landscape has changed significantly because of COVID-19, with the introduction of new VCAs (e.g. Google Meet, Microsoft Teams) and renewal of incumbent VCAs (e.g. Zoom, Bluejeans). Because these VCAs differ in their design choices and features, it is critical that we understand how these differences impact application performance and network consumption under different network conditions and call modalities. This has two-fold advantages: i) a comparative analysis can uncover best design practices useful for improving current and future VCAs, and ii) understanding VCA network consumption for different contexts can be useful for ISPs in network provisioning and for telecommunication policy makers in defining benchmarks for Broadband access.   




%We find differences in VCA behavior under same static network conditions, such as different maximum (or nominal) bitrate (0.8 Mbps-1.9 Mbps) and inefficient network utilization for some VCAs under constrained links (e.g., 50\% for Meet). Next, we study how each VCA responds to dynamic network conditions by introducing (1) temporary bandwidth drops or (2) a competing application (e.g. Netflix, TCP iPerf3) on the same link. All require at least 20 seconds to return to their nominal sending rate following a temporary uplink constraint to 0.25Mbps. Further, Microsoft Teams is especially slow to recover from drop in downlink bandwidth because they do not do congestion control from the cloud. When competing with other applications on a constrained link, Zoom aggressively consumes available bandwidth, while Meet and Teams back off at low bandwidths. Finally, we investigate how usage modality (e.g. number of participants, speaker vs. gallery mode) affect consumption. Increasing the number of participants may reduce a user's network utilization. We find that a user's sending bitrate at least doubles when their video is pinned by all other participants. 

%All VCAs require at least 20 seconds to return to their nominal sending rate following a temporary uplink constraint to 0.25Mbps. Further, Microsoft Teams is especially slow to recover from drop in downlink bandwidth because they do not do congestion control from the cloud. When competing with other applications on a constrained link, Zoom aggressively can consume over 75\% of the available bandwidth when competing with Meet and Teams. 
