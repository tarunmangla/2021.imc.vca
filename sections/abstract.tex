\begin{abstract}
Video conferencing applications (VCAs) have become a critical Internet
application during the COVID-19 pandemic, as users worldwide now rely on them for
work, school, and telehealth. 
It is thus increasingly important to understand the resource requirements of different
VCAs and how they perform under different network conditions, including:
\rev{how much ``speed'' (upstream and downstream throughput) a VCA needs to support high quality of experience}{how do application-layer performance metrics (e.g., resolution or frames per second) vary under different link capacity}; how
VCAs perform under temporary reductions in available capacity; how they compete
with themselves, with each other, and with
other applications; and how usage modality (e.g., gallery vs. speaker mode)
affects utilization.  We study three modern VCAs: Zoom, Google Meet,
and Microsoft Teams.   Answers to these questions differ substantially
depending on VCA.  First, the average utilization on an unconstrained
link varies between 0.8~Mbps and 1.9 Mbps.  Given temporary reduction of
capacity, some VCAs can take as long as 50 seconds to recover to steady state.  Differences in proprietary
congestion control algorithms also result in unfair bandwidth allocations: in
constrained bandwidth settings, one Zoom video conference can consume more than 75\%
of the available bandwidth when competing with another VCA (e.g., Meet, Teams).  For some VCAs, client utilization can
decrease as the number of participants increases, due to the reduced video
resolution of each participant's video stream given a larger number of
    participants. Finally, one participant's viewing mode
    (e.g., pinning a speaker) can affect the
upstream utilization of other participants.
\end{abstract}
