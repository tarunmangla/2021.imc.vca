\section{Related Work}\label{sec:related}

\paragraph{VCA measurement}: Some of the early VCA measurement work has focused on uncovering the design of the Skype focusing on its streaming protocols~\cite{baset2004analysis}, architecture~\cite{guha2005experimental}, traffic characterization~\cite{bonfiglio2008tracking}, and application performance~\cite{hossfeld2008analysis}. More recent work has included other VCAs and streaming contexts~\cite{xu2012video, yu2014can, azfar2016android}. Xu et al.~\cite{xu2012video} use controlled experiments to study the design and performance of Google+, iChat, and Skype. The work is further extended to include performance of the three services on mobile video calls~\cite{yu2014can}. 

Closest to our work is work by Jansen et al.~\cite{jansen2018performance}, Nistico et. al~\cite{nistico2020comparative}, and Chang et al. Jansen et al. evaluate WebRTC performance using their custom VCA under controlled network conditions~\cite{jansen2018performance}. Emulating similar network conditions, we consider performance of commercial and more recent VCAs that widely used for education and work. Even between tested VCAs using WebRTC, namely \meet and \teamsbrowser, we find significant performance differences, likely due to different design parameters (e.g., codecs, default bitrates). Nistico et al.~\cite{nistico2020comparative} consider a wider range of recent VCAs, focusing on their design differences including protocols and architecture. Our work provides a complementary performance analysis for a subset of the VCAs studied by them. We use the insights from their work to explain the differences among VCAs' network utilization and performance under similar streaming contexts. \rev{}{Finally, Chang et al.~\cite{chang2021} study the effects of geography, client device, and number of users on streaming experience in \meet, \teams, 
and Webex by emulating calls on the cloud. Our controlled experiments provide a complimentary method to evaluate the application performance. We also focus on response to disruptions and fairness of the VCAs.}

\paragraph{VCA congestion control}: Several congestion control algorithms have been proposed for VCAs. These algorithms rely on a variety of signals such as loss~\cite{handley2003tcp}, delay~\cite{carlucci2016analysis}, and even VCA performance metrics~\cite{singh2012rate} for rate control. For instance, Google Congestion Control~\cite{carlucci2016analysis}, also implemented in WebRTC, uses one-way delay gradient for adjusting the sender rate while SCReAM~\cite{johansson2015self} relies on both loss and delay along with TCP-like self-clocking. %Salsify~\cite{fouladi2018salsify}, proposes a new approach to congestion control through a integration between congestion control and video encoding. 
While the VCAs may use one or more of these variants, the exact implementation of the algorithm and parameter values vary and is proprietary. In this work, we study the efficacy of the VCA congestion control in the case of transient interruptions and background applications. A recent study by Sander et al. evaluates \zoom's congestion control along these dimensions~\cite{sandervideo}. Our work observes similar results for \zoom and also analyses more VCAs, including their fairness to each other and other popular internet applications, namely YouTube (QUIC-based) and Netflix (TCP-based). 


\begin{comment}
\paragraph{Performance of Internet applications}
There is also related work on measuring other applications over the Internet including video streaming~\cite{}, web browsing~\cite{}, and online gaming~\cite{}. Our work focuses on VCAs which are characterized by their strict latency requirements and upstream link utilization.  
\end{comment}
